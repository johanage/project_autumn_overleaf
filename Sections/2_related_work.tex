\section{Other scaling approaches of the total radiated power}
\label{sec:related_work}

\subsection{P.H. Rebut, B.J. Green - Effect on impurity radiation on tokamak equilibrium}

\cite{rebut1977effect} derive an energy balance equation including the impurity contribution. The terms given by the authors is:
\begin{equation}
    \sum_Z n_e^2 \alpha_Z R_Z
\end{equation}
where $n_e$ is the electron density, $\alpha_Z$ is the impurity fraction and $\alpha_Z n_e R_Z$ is the radiation power for impurities with density $\alpha_Z n_e$ where $R_Z$ is the sum of the photon emission coefficients (PEC) over all charge states, i.e. the fractional abundance. $R_Z$ is dependent on temperature; $R_Z = R_Z(T)$.

The presence of impurities causes bremsstrahlung in the core, line radiation and recombination radiation according to \cite{rebut1977effect}. High $Z$-materials will typically not be fully ionized, but radiates strongly and vice versa for low-$Z$ materials. The trade-off is the level of radiation against the level if ionization and the tolerance of contamination. For high-$Z$ materials the level of contamination is lower than for low-$Z$ materials.

\subsection{G.F. Matthews et al. - Scaling radiative plasmas to ITER}

The starting point of the scaling analysis done by \cite{matthews1997scaling} is the simple assumption that the radiation comes from a plasma volume $V_{rad}$ where the electron and impurity densities are $n_{e,rad}$ and $n_{Z, rad}$. The power can then be written on the form as shown in Eq. \ref{eq:matthews_prad_eq1}:
\begin{equation}
    P_{rad} = V_{rad}n_{e,rad}n_{Z,rad}L_z + P_{CX}
\label{eq:matthews_prad_eq1}
\end{equation}

where $P_{CX}$ is the power loss due to charge exchange and $L_z$ is the average radiated power coefficient in the radiating volume. The database used by \cite{matthews1997scaling} uses total radiated power from the main chamber bolometer systems. These systems are screened from divertor charge exchange, thus $P_{CX}=0$ is assumed.

For high $Z$ radiating impurities a reasonable assumption is that the radiated power is coming from a uniform shell of thickness $\Delta$ where $V_{rad} = \Delta S$ for a plasma surface $S$. Using this assumption an expression can be derived giving a relationship between the impurity and electron densities in the radiating zone and line averaged values. This is given in Eq. \ref{eq:matthews_zeff_deriv_1}:
\begin{equation}
    f_z = \frac{P_{rad}}{S \Delta C_n C_z \Bar{n_e}^2 L_z} = \frac{\Bar{n_Z}}{\Bar{n_e}}
\label{eq:matthews_zeff_deriv_1}
\end{equation}
where $C_n = \frac{n_{e,rad}}{\Bar{n_{e}}}$, $C_Z = \frac{n_{Z,rad}}{\Bar{n_{e}}}$. This gives a corresponding equation for $Z_{eff}$ assuming one main radiator of charge state $Z$:
\begin{equation}
    Z_{eff}\approx 1 + Z(Z-1)f_z = 1 + \frac{Z(Z-1)P_{rad}}{S\Delta C_n C_Z \Bar{n_e}^2 L_z}.
\label{eq:matthews_zeff_eq3}
\end{equation}

However, the database used by \cite{matthews1997scaling} does not contain any information about $\Delta, C_n, C_Z, L_Z$, so a non-linear regression was applied to the following functional form:
\begin{equation}
    Z_{eff} = 1 + \frac{\alpha P_{rad} Z^{\delta}}{S^{\beta}\Bar{n_e}^{\gamma}},
\label{eq:matthews_zeff_regr_func_form}
\end{equation}

where $\alpha, \beta, \delta$ and $\gamma$ was determined giving rise to the following result:
\begin{equation}
    Z_{eff} = 1 + \frac{5.6(\pm 0.7)P_{rad}Z^{0.19\pm 0.05}}{S^{1.03 \pm 0.02}\Bar{n_e}^{1.95\pm 0.04}}
\label{eq:matthews_zeff_regression_result}
\end{equation}

where $P_{rad}$ is in $\mathrm{MW}$, $S$ in $\mathrm{m^2}$ and $\Bar{n_e}$ in units of $10^{20}\mathrm{m^{-3}}$.

Expressed for $P_{rad}$ the model is then:
\begin{equation}
    P_{rad} = \alpha\frac{ S^{\beta} \Bar{n}_e (Z_{eff}-1)}{Z^{\delta}}
\end{equation}

However interesting, this model is suitable for machines where the magnetic geometry allows for a reasonable calculation of the radiation surface area $S$ which is hard to compute for stellarators which have complex magnetic geometries. A crude assumption often made is that the radiation surface area is the same as the surface of the a torus with minor and major radius equal to that of the machine, i.e. $4\pi^2 r R$. As demonstrated by \cite{effenberg2019first} this is not necessarily a valid assumption.

\subsection{Goldston et al. - A new scaling for divertor detachment}

\cite{goldston2017new} evaluates the upstream heat flux which can be dissipated by impurity radiation for divertor detachment:
\begin{equation}
    q_{\parallel, det} = n_{e,sep} T_{e,sep} \sqrt{2\int_{T_{e,det}}^{T_{e,sep}}  F_z \kappa_0 T_e^{0.5}L_z(T_e)dT_e}
\label{eq:goldston_qpar_detachment}
\end{equation}

where $q_{\parallel, det}$ is the heat flux entering the SOL for a detached plasma, $n_{e,sep},T_{e,sep}$ is the electron density and temperature at the separatrix, $T_{e,det}$ is the temperature at the point of detachment, $F_z = c_z \kappa_z$ where $\kappa_z \equiv \kappa_0 T_e^{2.5}$ for $Z = 1$ and $c_z \equiv \frac{n_z}{n_e}$. Introducing impurities leads to the need for correcting $\kappa_z$ since the former no longer holds for high $Z$-species. Braginskii provides this correction for discrete values, and using a fit \cite{goldston2017new} finds a fit for $\kappa_z = (0.672 + 0.076 Z_{eff}^{0.5} + 0.252 Z_{eff})^{-1}$ accurate up to $1\%$ precision. To find the impurity concentration $c_Z$ for a single dominant impurity the following relation can be derived $Z_{eff} = 1 + c_Z (Z^2-Z) \implies c_Z = \frac{Z_{eff}-1}{Z^2-Z}$. 

To evaluate the integral in Eq. \ref{eq:goldston_qpar_detachment} the integration boundaries needs to be expressed in reasonable parameters. Using the two-point model the temperature at the separatrix can be, according to \cite{goldston2017new},  expressed as:
\begin{align}
    T_{e,sep} &= \left( \frac{7}{2}\frac{q_{\parallel}L}{\kappa_Z \kappa_0} \right)^{2/7}\nonumber \\
    &= \left( \frac{7}{2}\frac{q_{\parallel}l_{\parallel}^{*}\pi q_{cyl}R}{\kappa_Z \kappa_0} \right)^{2/7}
\end{align}

where the $L$ is the connection length, $l_{\parallel}^{*}\pi q_{cyl}R$ is the estimate of the connection length for a given magnetic configuration for a tokamak where $l_{\parallel}^{*}$ is the correction to the connection length for a magnetic configuration different from the standard magnetic configuration. $q_{cyl}$ is the safety factor after the corrected conductivity of the diluted plasma (plasma with impurities). Thus all the ingredients for evaluating Eq. \ref{eq:goldston_qpar_detachment} is found and the radiation loss for detached cases following the respective definition of detachment from the paper can be evaluated. This gives an interesting approach to the scaling problem, but is unfortunately restricted for detached cases and does not hold for a general scaling for the total radiated power. \cite{goldston2017new} also exploits the dimensionality reduction of the problem to derive the scaling with global plasma parameters important for tokamaks which is not applicable for the W7-X database.

\subsection{Perseo et al. - 2D measurements of parallel counter-streaming flows in the W7-X scrape-off layer for attached and detached plasmas}

\cite{perseo20212d} presents a linear scaling for the total radiated power with $I_{C^{2+}}$ as given in Eq. \ref{eq:scaling_perseo_ic2+}:
\begin{equation}
\label{eq:scaling_perseo_ic2+}
    P_{rad} = a\cdot I_{C^{2+}} + b
\end{equation}

They plot the total radiated power measured by W7-X's bolometry system against the $I_{C^{2+}}$ measured by coherence imaging spectroscopy and observe a linear relationship with the radiated power. Furthermore they fit the $I_{C^{2+}}$ data with a linear least square model and get $a = 0.08, b = -0.43$ and a correlation coefficient $r = 0.98$ which strongly strengthens argument for the observed relationship. Furthermore \cite{perseo20212d} also investigates an empirical model of the total radiated power given by Eq. \ref{eq:scaling_perseo_nbare}:

\begin{equation}
\label{eq:scaling_perseo_nbare}
    P_{rad} = \frac{1}{7}(Z_{eff}-1)S \Bar{n}_e^{\gamma}
\end{equation}

where $S$ represents the surface of the radiating plasma volume and is estimated by $4\pi^2 r R$ and the only free regression parameter $\gamma$ is set for the line integrated density. Note that the dimensions of the left hand side is not equal to the right hand side which implies that the temperature dependent radiation contribution is set to a constant. The observation is that the fit shows a enough correlation to argue for a link between $I_{C^{2+}}$ and the line integrated density when considering their shared correlation to the total radiated power for $\gamma \geq 2$. $Z_{eff}$ is used to represent the impurity fraction.

\subsection{D. Zhang et al. - Plasma radiation behavior approaching high-radiation scenarios in W7-X }

\cite{zhang2021plasma} presents a similar scaling for $P_{rad}$ given in Eq. \ref{eq:zhang_model}:
\begin{equation}
\label{eq:zhang_model}
    P_{rad} = V_p \Bar{n}_e^2(Z_{eff} - 1)\alpha_R
\end{equation}

where $V_p$ is the plasma volume, $Z_{eff}$ is the effective ion charge, and $\alpha_R$ is a function of temperature related to the input power $P_{Heat}$, impurities and the impurity transport. A model for the radiated power fraction is also presented given en Eq.
\begin{equation}
    f_{rad} = V_p \Bar{n}_e^2(Z_{eff} - 1)\alpha_f
\end{equation}

where $\alpha_f = \frac{\alpha_R}{P_{Heat}}$. This definition is similar to the definition of radiated power fraction in this paper. The result of the computation using the database (same as for this paper) for  $\alpha_f \propto P_{Heat}^{-\beta}$ is $\beta \approx 1$. 