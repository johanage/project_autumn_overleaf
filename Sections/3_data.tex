\section{Data}
\label{sec:data}

\subsection{Experimental data}

Experimental data from IPP's Juice database has been used for the radiated power scaling. It is a database of global plasma parameters for every discharge in the first campaign where every observable have been averaged over $200 \mathrm{ms}$ of the discharge for denoising purposes. The observables in the database comes from the W7X archive which is a web based service for accessing data measured by the diagnostics.

\subsection{Motivation for observables used in the regression model}

\subsection{Dimension analysis for data used in regression model}

\begin{table}[H]
    \centering
    \begin{tabular}{|c|c|c|}
        \hline
        Observable & SI & Diagnostic dimension  \\
        \hline
        $P_{rad}$ & $\mathrm{W}$ & $\mathrm{MW}$ \\ \hline
        $P_{Heat}$ & $\mathrm{W}$ & $\mathrm{MW}$ \\ \hline
        $\Bar{n}_e$ & $\mathrm{m^{-3}}$ & $\mathrm{m^{-3}}$ \\\hline
        $Z_{eff}$ & Dimensionless & Dimensionless  \\\hline
        $p_{neutral}$ & $\mathrm{mBar}$ & $\mathrm{mBar}$ \\\hline
        $I_{C^{2+}}$ & $\mathrm{W m^{-3}}$ & $\mathrm{pixel count/ms}$  \\
        \hline
    \end{tabular}
    \caption{Dimensions of observables used in the regression model. SI dimensions are compared with the dimensions given by the diagnostic.}
    \label{tab:my_label}
\end{table}

\subsection{\texorpdfstring{$P_{rad}$}{TEXT}}

The radiated power loss $P_{rad}$ has been measure by the bolometer system which consists of one horizontally and one vertically viewing cameras by 32 and 43 collimated blackened detectors. The triangular cross-section of the reactor is covered with the lines of sight for various magnetic configurations. Having measurements along a line of sight enables you to integrate along the line of sight to acquire line-integrated signals such as radiate power loss $P_{rad}$ from the plasma.

\begin{table}[H]
    \centering
    \begin{tabular}{|c|c|}
        \hline
        Spectral range & $2\mathrm{eV} - 5\mathrm{eV}$  \\ \hline
        Detectors & Resistive metal\\ \hline
        Absorbers & $5\mathrm{\mu m}$ gold\\ \hline
        Coating & $50\mathrm{nm}$ - C\\ \hline
        Filters &  $6\mathrm{\mu m} - 12\mathrm{\mu m}$ - Beryllium \\
        \hline
    \end{tabular}
    \caption{The coating is for improving the absorption coefficient in the visible spectral range. The properties of the bolometer systems measuring the radiated power $P_{rad}$. The filters enables those to obtain some information about the soft X - ray radiation ($>800\mathrm{eV}$). The properties of the secondary array of bolometer systems measuring the radiated power $P_{rad}$ form the high-$Z$ impurities.}
    \label{tab:bolometer_system_properties}
\end{table}

\subsubsection{$P_{ECRH}$}

To get the radiated power fraction the radiated power needs to be divided by the power entering the SOL $P_{SOL}$. Assuming that the losses from the main plasma is small the power entering the SOL can be approximated to be the input power of the machine $P_{in}$. In W7X this is mainly $P_{ECRH}$ - the electron cyclotron heating, but Neutral Beam injection (NBI) is included for the relevant campaign in the approximation of the input power of the SOL.

\subsubsection{$\Bar{n}_e$}

The line integrated electron density $\Bar{n}_e$ is evaluated along the line of sight (LOS) by the Integral Electron Density Dispersion Interferometer (IEDDI). The diagnostic exploits the change in the refractive index of the plasma which is only proportional to the wavelength of the laser light and the local density of free electrons to derive the density. The change in refractive index is governed by the electron density for the laser wavelength used. Since it is a dispersion interferometer is applies a Frequency Doubling Crystal (FDC) which mitigates any vibration induced phase shifts. It is used as the density feedback-control measurement since it is only sensitive to the free electron density which corresponds to the plasma density. 

\begin{table}[H]
    \centering
    \begin{tabular}{|c|c|}
        \hline
        Location & Laser path: AEZ31, AET31\\ \hline
        Location & Toroidal angle: 170,5$\circ$\\ \hline
        Measured range & $10^{17} - 10^{21} \mathrm{m^{-2}}$\\ \hline
        Temporal resolution & $\leq 50 \mathrm{kHz}$\\ \hline
        Radial resoultion & None (Single channel)\\ \hline
        Channels & Single channel  \\ \hline
        Power & 20 $\mathrm{W}$\\ \hline
        Laser type &  $\mathrm{CO_2}$\\ \hline
        Wavelength  & 10.6 $\mathrm{\mu m}$\\ \hline
        Harmonicity ($\mathrm{AfGaSe_2}$ frequency doubling crystal) & Second harmonic \\ \hline
        Interference wavelength & 5.3 $\mathrm{\mu m}$\\
        \hline
    \end{tabular}
    \caption{Caption}
    \label{tab:my_label}
\end{table}

The detected phase difference of the signal measured can be shown to be independent of geometrical path length along the beam path. This means that the vibrations of the optical components will not affect the phase signal. The IEDDI has the same LOS as the Thomson scattering system and a cross calibration can therefore be performed making the measurement more robust. The magnetic configuration affects the path length through the plasma so a good average value is used for the single-pass $\sim 133 \mathrm{cm}$.

\subsubsection{$Z_{eff}$}

The line-averaged effective ion charge number is inferred from bremsstrahlung spectrometer data using a Bayesian model based on the Minerva framework which infers profiles with Gaussian processes [\cite{pavone2019measurements}].

\subsubsection{$p_{neutral}$}

The neutral pressure gauges  measures the neutral gas pressure in $\sim 18$ toroidally and poloidally distributed positions around the torus. The gauges in use are hot-cathode ionization gauges, optimized for operation in strong magnetic field and for long pulse operation. 

\begin{table}[H]
    \centering
    \begin{tabular}{|c|c|}
    \hline
    Locations &  Main chamber, divertor, pumps\\ \hline
    Operational limits and availability     & Routine operation\\ \hline
    Mode of operation & Single-shot, continuous, scanning\\ \hline
    Spatial resolution & Real or k-space, range or single value\\ \hline
    Frequency resolution & \\ \hline
    radial coverage & \\ \hline
    Frequency resolution & \\ \hline
    Calibration & Absolutely calibrated neutral gas pressure\\ \hline
    Data availability & Data available in the archive\\ 
    \hline
    \end{tabular}
    \caption{Operational details on the measurement of neutral pressure from the neutral pressure gauges.}
    \label{tab:p_neutral_data_info}
\end{table}

\subsubsection{$I_{C^{2+}}$}

The intensity of the emission from impurity and neutral radiation is measure by the Coherence Imaging Spectroscopy (CIS). It is a camera-based interferometric system which measures the Doppler particle flow velocities and temperature associated with a selection of visible emission lines form the plasma. The interference pattern is generated by birefringent crystals and overlaid on camera images of the plasma visible radiation, allowing 2D LOS integrated measurements. The choice of impurities and neutral is selected by narrowband filters. It is also possible to select single charged state of single particle species.

\begin{table}[H]
    \centering
    \begin{tabularx}{0.95\textwidth}{|X|X|}
        \hline
        Measured quantity & Emission line intensity, particle flow velocity in SOL, particle temperature in SOL (under development)\\ \hline
        Narrowband filter wavelength & $\sim 2 \mathrm{nm}$ \\ \hline
        Location & AEA21 port - toroidal view from module 2 into module 1
        AEF30 port - vertical view on lower divertor module 3\\ \hline
        Operational limits and availability & Routinely operational, species under investigation may vary\\ \hline
        Mode of operation & continuous\\ \hline
        Resolution radiation intensity & depending on particle species\\
        Resolution temperature & Under development\\ \hline
        Resolution flow velocity & from 2-3 $\mathrm{km / s}$ to 50 $\mathrm{km / s}$\\ \hline
        Calibration radiation intensity & no absolute calibration available, only relative quantities\\ \hline
        Calibration flow velocity & absolute calibration with tunable laser\\ \hline
        Calibration temperature & under development\\ \hline
        Data availability & Raw data available at the end of each operational day, processed data under request\\ \hline
    \end{tabularx}
    \caption{The properties of the Coherence Imaging Spectroscopy diagnostic.}
    \label{tab:CIS_properties}
\end{table}