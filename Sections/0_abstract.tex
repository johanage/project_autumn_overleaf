\begin{tcolorbox}
Knowing the magnitude of machine inputs to optimize the power balance of fusion reactors is crucial for harnessing the fusion power for reactors with divertor configurations. The current challenge is to avoid large peaks in the heat flux and spread out the power exhaust on the divertor plates without ending up below the acquired threshold for a stable fusion reaction. Understanding the processes at play to achieve this is difficult because the correlation of these processes is extremely complex. Therefore simplifications have to be made with reasonable assumptions in order to come with conclusions on leading behaviour. Simplified models, each relevant for a specific regime, can be used to understand the main underlying physics. More sophisticated models are used for simulating the process in a fusion reactor as done in the EMC3 EIRENE code. The latter combines fluid, kinetic and stochastic regimes and outputs the scalar fields of the quantities of interest based on Monte Carlo integration. Even though EMC3 EIRENE are a based on a sophisticated model, simulations of the fusion reaction has to be verified through comparison with experimental data. 

\vspace{0.5cm}

In this report experimental data are used to verify EMC3-EIRENE simulations with a choice of reasonable parameter inputs by comparison, using synthetic diagnostics. It also addresses for which regimes the simplified models can be used and for which regimes it breaks down. The result from the latter can be used to assess the validity of looking at power balance along magnetic field lines and if it can be extended to make conclusions about the power distribution in the machine. 

\vspace{0.5cm}

Results:

\vspace{0.5cm}

Conclusion:
\end{tcolorbox}