\section{Introduction}
\label{sec:introduction}
\begin{multicols}{2}

\subsection{Terms}
\begin{enumerate}
    \item \textbf{ECRH} - Electron Cyclotron Resonance Heating 
    \item \textbf{PF} - Private flux
    \item \textbf{LTS} - Lower target shadowed
    \item \textbf{UTS} - Upper target shadowed
    \item \textbf{IS} - Inner shadowed region
    \item \textbf{OS} - Outer shadowed region
    \item \textbf{Connection length} - $L$ is called the connection length, and the distance along $\bm B$ in the SOL between two points of contact with the solid surface is $2L$
    \item \textbf{X-point} - the point where the open field lines bounding the separatrix overlaps.
    \item \textbf{MPS} - magnetic pre-sheath
    \item \textbf{PSI} - plasma surface interaction. Interaction between plasma and a solid (machine parts).
    \item \textbf{Sputtering} - the process where plasma particles collides with the surface of the reactor dislodging atoms from the lattice. Can lead to erosion and eventually contamination of the plasma.
    \item \textbf{Debye sheath} - a thin net charge boundary layer at the interface between the solid and the plasma. Plays a central role in establishing the temperature, density and other properties of the plasma.
    \item \textbf{Plasma wetted area} - the "small" area of the target where the power deposited by the electron and ion impact occurs. To small of an area can cause to overheating and melting of the target. In other words where the solid surface is in contact with significant plasma fluxes.
    \item \textbf{Relfectivity} - the fraction of neutrals incident on the plasma that gets reflected back due to charge exchange.
\end{enumerate}
\end{multicols}

\newpage
\subsubsection{Stellarator}

A stellarator is magnetic confinement device made for realising nuclear fusion. The separation from the tokamaks is that the rotational transform is generated by external field coils - not plasma currents. 

\subsubsection{Divertor}

A divertor configuration is a magnetic field configuration where the confined plasma is separated from the outside world by a separatrix. The main purpose of introducing a divertor as a replacement for the limiter (physical walls to contain the plasma) was to reduce the amount of impurities. There is a difference between the divertor configuration for the tokamaks which use a standard divertor configuration with one or two X-points and island divertors for stellarators. A magnetic island is a closed magnetic flux tube with a toroidal topology which is bounded by a separatrix. 

\subsubsection{Separatrix}

A separatrix is the boundary separating two modes of behaviour (phases) in a dynamical system. In the context of magnetically confined fusion it refers to the trajectories of the magnetic field lines. It is the boundary between the inside and outside of a magnetic island, separating between closed and open magnetic field lines in a divertor configuration. 

\subsubsection{Last Closed Magnetic Surface (LCMS)}

The last closed flux surface is the boundary between the confined plasma and the external part of the plasma that interacts with the wall. 
\subsubsection{Scrape-off Layer (SOL)}

The scrape-off layer defines the plasma region governed by open field lines which begins and ends at material surfaces. It is the region outside the LCFS in the divertor configuration this region is outside the separatrix. The fact that the scrape-off layer is governed by open magnetic field lines strongly differs the transport from the transport in the confined plasma region because it is predominantly convective rather than diffusive. The density in the SOL typically decays exponentially away from the LCFS. However, a detailed description of the transport phenomena is needed because intermittency is strong in the SOL region. 

\subsection{Sputtering}

Sputtering is the term related to when the particles from the plasma collides with the reactor wall and breaks free a part of the wall thus introducing impurities to the plasma. Typically the impurities from sputtering is carbon atoms.

\subsection{Motivation for characterising the radiated power}

Future fusion reactors needs to be run for long discharges over several decades to be considered as an economically efficient way of generating energy. This means that the machine has to be durable enough to widthstand the extreme conditions which are needed for fusion reactions over a long time. The maximum target heat load is one of the limiting factors of a reactors life time and even also for the making fusion possible. Since the material of the divertor plates only tolerate a limited heat load the need for mechanisms to cool the plasma has become significant. The main cooling mechanism is radiation due to ionisation of impurities in the plasma. Exploiting the larger ionisation energy of species with a higher $Z$ number the plasma can potentially be cooled enough for acceptable heat loads on the divertor plates. This is a process which would be beneficial to control such that the amount of cooling of the divertor plates can be controlled. However, the physics and how the radiation from impurities scales with global plasma parameters is still unclear and needs to be investigated further. Understanding the physics of such a phenomena is not straight forward since it is governed by transport processes which makes the scaling inherently dependent on local parameters rather than global parameters. Unfortunately a scaling with local parameters is not easy as observables measured by the diagnostics of the W7X are mainly global parameters. 

Ideally such a scaling law which is based on controllable parameters would not only help the understanding of the physics of the radiation loss in the SOL, but can also potentially be used as a base for a model for feedback or feed-forward systems. For tokamaks a feedback system for the radiation loss would mean steady state operation for energy, assuming the radiation loss to be the main power loss. In stellarators such as W7X which is built for steady state operation a feed-forward system with a proper scaling law would enable the experiment to infer a desired amount of radiation loss for a chosen set of global plasma parameters.

\newpage

\subsection{Fluid equations}

Before introducing the two-point model the relevant equations should be derived from the general fluid equations to understand when the model is applicable. The fluid equations can be summed up from three basic physical principles for systems in stationary phase:
\begin{enumerate}
    \item Continuity.
    \item Momentum balance. 
    \item Energy balance.
\end{enumerate}

For charged particles the general equations for continuity, momentum and energy conservation are:
\begin{align}
   \frac{\partial n_p}{\partial t} +  \nabla \cdot (n_pv) = 0\\
    n_p(\frac{\partial \bm{u}}{\partial t} + \bm{u}\cdot \nabla\bm{u}) = \bm{F} - \nabla\bm{p}\\
    n_p\frac{d \epsilon}{dt} = -\nabla \bm{q} - \bm{p}\:\nabla\bm{u}
\end{align}

where $n_p$ is the density of particle of species $p$, $\bm{p}$ is the second order pressure tensor, $\bm{u}$ is the bulk fluid velocity. Note that if $\bm{p} \equiv p\delta_{ij}$ where the off-diagonal elements are zero if the flow of the fluid is isotropic. However, this is normally not the case for plasma fluids for many reasons, one of them being that it consist of different species of charged and neutral particles. It is usual to separate between parallel and perpendicular flow of the plasma, where the parallel flow is along magnetic field lines, because plasma tends to follow these. In addition the steady state in fusion reactors is modelled with sources due to ionisation, recombination and other transport phenomena. So for the plasma-friendly treatment of fluid equations we say that the differentiable operator is split in a parallel and perpendicular part as given in Eq. \ref{eq:def_nabla_operator}:
\begin{equation}
    \nabla \equiv \nabla_{\parallel} + \nabla_{\perp}
\label{eq:def_nabla_operator}
\end{equation}

which gives the following fluid equations for continuity, momentum and energy conservation:
\begin{align}
    \nabla \cdot (n_i V_{i,\parallel}\bm b - D \nabla_{\perp}n_i) &= S_p\\
    \nabla \cdot (m_i n_i V_{i,\parallel}^2 \bm b - \eta_{\parallel}\nabla_{\parallel}V_{i,\parallel} - m_iV_{i,\parallel}D\nabla_{\perp}n_i - \eta_{\perp}\nabla_{\perp}V_{i,\parallel}) &= -\nabla_{\parallel} p + S_m\\
    \nabla \cdot (\frac{5}{2}n_eT_eV_{i,\parallel}\bm b - \kappa_e\nabla_{\parallel}T_e - \frac{5}{2}T_eD\nabla_{\perp}n_e-\chi_e n_e\nabla_{\perp}T_e) &= - k(T_e - T_i) + S_{ee} + S_{imp}\\
    \nabla \cdot (\frac{5}{2}n_iT_iV_{i,\parallel}\bm b - \kappa_i\nabla_{\parallel}T_i - \frac{5}{2}T_iD\nabla_{\perp}n_i-\chi_i n_i\nabla_{\perp}T_i) &= + k(T_e - T_i) + S_{ei}
\end{align}

where $D$ is the diffusion coefficient where a subscript refers to the direction of the diffusion, $S_p$ is the source/sink of particles, $S_m$ refers to momentum loss, $\kappa_e, \kappa_i$ is the classical heat conduction coefficients for electrons and ions, $\chi_e, \chi_i$ is the cross-field transport coefficients for electrons and ions, $S_{ee}$ is the energy loss due to electron-electron collisions, $S_{imp}$ is the energy loss due to collisions between electrons and impurities and $S_{ei}$ is the source due to electron-ion collisions. Note that $q_{\parallel} = -\kappa_e (T_e)\nabla_{\parallel}T$.

This continuous 3D model is a very complicated starting point for expressing the radiation loss with global plasma parameters. Therefore a simplified model called the two-point model has been invented comparing the conditions between two spatial positions. This enables a interpretable analysis of scaling between global plasma parameters. However, the simplicity of the model comes at the price of a lot of assumptions which potentially limits the validity of the model. It is still relevant for the purposes of this paper since the assumptions could be good enough for the system on which they are applied.

\subsection{Two-point model}

\begin{figure}[H]
    \centering
    \includegraphics[scale=0.5]{Images/2point_model_figure.png}
    \caption{For purposes of simple modelling, the divertor SOL is ‘straightened out’ in a 1D-model along magnetic field lines. Since (calculated) parallel gradients are usually small at locations far from the targets, the precise location chosen to represent the ‘upstream’ point is not critical.}
    \label{fig:2pointModel}
\end{figure}

\begin{tcolorbox}
\textbf{2 points referred to:}
\begin{enumerate}
    \item u - upstream. Location halfway to targets.
    \item t - target. SOL is straightened out for easy modelling.
\end{enumerate}
\end{tcolorbox}

\begin{comment}
The transport equations along field lines are given by Eq. \ref{eq:balance_eqs}:
\begin{align}
    \frac{\partial }{\partial x}nV= -S_{n}\\
    \frac{\partial}{\partial x}[nm_iV^2 + T_e(I + r)] = S_{p}\\
    \frac{\partial}{\partial x}\left\{ q_e + nV\left[ \frac{1}{2}m_iV^2 + \frac{5}{2}nT_e(1 + r) \right] \right\} = S_{E}
    \label{eq:balance_eqs}
\end{align}

where $S_n, S_p, S_E$ are the particle, momentum and energy sources associated with neutral recycling; $n$ is the particle density; $T_e$ is the electron temperature; $V$ is the fluid flow speed toward the plate; $r$ is the ion-to-electron ratio; $m_i$ is the ion mass and $q_e$ is the electron heat conduction $(k_BT_e^{5/2})$.
\end{comment}

The neglections are:
\begin{enumerate}
    \item Ion heat conduction since it is much smaller than electron heat conduction.
    \item Cross-field transport effects.
    \item Impurity radiation losses are neglected.
\end{enumerate}


The simple two point model describes:
\begin{enumerate}
    \item Particle balance - assumes that neutrals recycling from targets are all ionized in a thin layer immediately in front of the target. Further, a neutral which resulted from an ion impacting the target while travelling along a particular magnetic field line is assumed to be re-ionized on that same field line. In steady state, therefore, each flux tube has its own, highly localized particle balance with the same particles just recycling over and over, spending part of their time as ions and part of their time as neutrals.
    \item Pressure balance. It is assumed that there is no friction between the plasma flow in the thin ionization region and the target and no viscous effects. Thus throughout the entire length of each SOL flux tube:
    \begin{equation}
        p + nmv^2 = \textit{constant}
    \end{equation}
    assuming $T_e = T_i$ the static pressure is $p = nk(T_e + T_i) = 2nkT$. The dynamic pressure is given by $p = nmv^2$, thus giving the following relation between the upstream and target location:
    \begin{equation}\label{eq:upstream_target_relation}
        n_t(2kT_t + mv_t^2) = 2 n_u k T_u \implies 2n_tT_t = n_uT_u
    \end{equation}
    using $v_t = c_t = \left(\frac{2kT_t}{m_i}\right)^{1/2}$, coming from the Bohm condition, meaning the particle velocity entering the sheath is the sound velocity.
    \item Power balance. Due to $v=0$ along the SOL, heat convection and the parallel power flux density $q_{\parallel}[\mathrm{
    Wm^{-2}}]$ is carried out by conduction. If it is assumed that the parallel heat conduction $q_{\parallel}$ enters entirely at the upstream location and removed at the target at a distance $\mathrm{L}$ the following equation holds:
    \begin{equation}
        T_u^{7/2} = T_t^{7/2} + \frac{7}{2}q_{\parallel}\frac{L}{\kappa_{0e}},
    \label{eq:stangeby_5.5}
    \end{equation}
    where the electron parallel conductivity coefficient $\kappa_{0e}$ has been used under the assumption that electrons and ions are thermally coupled, neglecting parallel ion heat conductivity (comparably small to the electron heat conductivity). Further assumptions:
    \begin{enumerate}
        \item No volumetric power sources or sinks in the flux tube.
        \item The ionization region is thin, thus the temperature change over the ionization region is ignored $\implies T_t$ from Eq. \ref{eq:stangeby_5.5} is the temperature at the target sheath edge. This leads to the following equation for $q_{\parallel}$:
        \begin{equation}
        q_{\parallel} = q_t = \gamma n_t k T_t c_{st},
        \label{eq:stangeby_5.6}    
        \end{equation}
        where the $q_t$ is the heat flux density entering the sheath, $\gamma$ is the sheath heat transmission coefficient and $\gamma \approx 7$
    \end{enumerate}
\end{enumerate}

\subsection{Pressure balance - The Bohm boundary condition}



\subsection{Summarized}

\begin{tcolorbox}
For the two-point model we have following three equations with three unknowns $n_t, T_t, T_u$:
\begin{align}
    2n_t T_t = n_u T_u\\
    T_u^{7/2} = T_t^{7/2} + \frac{7}{2}\frac{q_{\parallel}L}{\kappa_{0e}}\\
    q_{\parallel} = \gamma n_t k T_t c_{st},
\end{align}
\label{eq:two_point_model_stangeby}
\end{tcolorbox}

where it seems natural to treat the following parameters as the controlled parameters:
\begin{enumerate}
    \item $n_u$
    \item $q_{\parallel}$
\end{enumerate}

and the independent variables specified as constants: $L, \gamma, \kappa_{0e}$.

These quantities can be related to the two principal control parameters for tokamak operators which is the input power $P_{in} [\mathrm{W}]$ and the main plasma density $\Bar{n_e}$. By assuming all ionization happens in the SOL and only diffusive radial transport in the main plasma, then $\Bar{n_e} = n_{LCDS}$. Even though pinches, inward drifts towards the main plasma, alters the ratio $\frac{n_{LCFS}}{\Bar{n_e}}$ we may treat $n_{LCFS}$ as being controlled or imposed as for $\Bar{n_e}$ for the purposes of edge analysis.

\subsubsection{Relation between $n_{LCFS}$ and $n_u$}

It is natural to develop the discussion towards the relation between $n_{LCFS}$ of the 1D radial analysis (Engelhardt model) and $n_u$. Let $n_u$ be the radially averaged values. However, $n$ is still not a constant value with the distance $s_{\parallel}$ along the SOL. This raises the question of which location along the SOL to equate the density $n_{LCFS}$ to. If the density dependent on the location along the SOL $n(s_{\parallel})$ varied greatly along the flux tube from X-point to the upstream end, where it interfaces the plasma, it would make it hard to choose the location for the density $n_{LCFS}$. 

TBC...

\subsubsection{The upstream temperature}

If the temperature is sufficient to assume $T_u^{7/2} \gg T_t^{7/2}$ Eqs. \ref{eq:two_point_model_stangeby} can be used to reduce the expression for the upstream temperature to Eq. \ref{eq:upstream_small_T_t}
\begin{equation}
    T_u \simeq \left( \frac{7}{2}\frac{q_{\parallel}L}{\kappa_{0e}} \right)^{(2/7)}
\label{eq:upstream_small_T_t}
\end{equation}
which assumes that all the power enters in the middle of the distance along a field line from target to target. 

\textbf{Independence of $n_u$}

For fully ionised plasmas neither electrical nor heat conductivity depend on the number of carriers which is completely opposite to the material properties for the other phases. Thus, for fully ionised plasmas the upstream temperature $T_u$ is independent of the density $n_u$.

\textbf{Dependence on $q_{\parallel}$}

The heat conductivity $K_{\parallel}$ of a fully ionised plasma is a strong function of temperature $K_{\parallel} \propto T^{5/2}$. A small change in the temperature  results in a large change in the heat conductivity which can lead to large changes in other factors. 

\subsubsection{Target temperature}

Combining equations from Eqs. \ref{eq:two_point_model_stangeby}:
\begin{align*}
    2n_t T_t &= n_u T_u\\
    q_{\parallel} &= \gamma n_t k T_t c_{st}\\
    q_{\parallel} &= \frac{1}{2}\gamma n_u T_u k \left(\frac{2kT_t}{m_i}\right)^{1/2}\\
    q_{\parallel}^2 &= \left( \frac{1}{2}\gamma n_u T_u k \right)^2 \left(\frac{2kT_t}{m_i}\right) = \frac{1}{2}\gamma^2n_u^2T_u^2 k^3 \frac{T_t}{m_i}\\
    T_t &= \frac{2q_{\parallel} m_i}{\gamma^2n_u^2T_u^2 k^3}
\end{align*}

and when expressing the temperature in $\mathrm{eV}$ this is the same as the target temperature being expressed by Eq. \ref{eq:T_t_two_point_model}:
\begin{equation}
    T_t = \frac{m_i}{2e}\frac{4q_{\parallel}^2}{\gamma^2e^2n_u^2T_u^2}.
\label{eq:T_t_two_point_model}
\end{equation}
where the units are $T[\mathrm{eV}, m_i[kg], q_{\parallel}[Wm^{2}], n[\mathrm{m^{-3}}]$ and $e$ the elementary charge.  In fact Eq. \ref{eq:T_t_two_point_model} holds whether the parallel heat convection is present or not. This follows from the sole assumption of pressure and power conservation. However, the target temperature still a function of $T_u$ and so Eq. \ref{eq:T_t_two_point_model} should be expressed as a function independent of the upstream temperature. This results in Eq. \ref{eq:T_t_two_point_model_indep_Tu} using the result from \ref{eq:upstream_small_T_t}:
\begin{equation}
     T_t \simeq \frac{m_i}{2e}\frac{4q_{\parallel}^2}{\gamma^2e^2n_u^2\left( \frac{7}{2}\frac{q_{\parallel}L}{\kappa_{0e}} \right)^{(4/7)}} \propto \frac{q_{\parallel}^{10/7}}{L^{4/7}n_u^2}.
\label{eq:T_t_two_point_model_indep_Tu}
\end{equation}

From Eq. \ref{eq:T_t_two_point_model_indep_Tu} it is clear that the target temperature is strongly dependent on the upstream density and the heat entering the SOL. The optimal scenario would be to have a relation to the input power $q_{\parallel}$ as weak as possible and the dependence on $n_u$ as strong as possible. The argument is that high input power is needed for fusion reactions, but it cannot exceed the tolerance of the material at the divertor plates. Since the target temperature is a strong function of the heat flux density entering the SOL which is closely related to the input power assuming that the main power loss comes from radiation and not alpha heating, it limits the input power. This again leads to a limit on the amount of fusion reactions (?). Since the target temperature is a strong function of upstream density, which can be approximated by the measurable $\Bar{n}_e$, it is favorable to operate fusion reactors at high densities which is also a requirement for high fusion power. Thus it is necessary to operate on high enough densities to avoid exceeding allowed target temperatures with respect to the material heat tolerance. 

The connection length of the SOL also affects the target temperature, but it is a weaker than linear relation thus making it a non practical way of reducing target temperature, sputtering, etc. 

\subsubsection{Target density}

Starting from the momentum equation the following expression for the target density can be derived, assuming $T_u^{7/2} \simeq \frac{7}{2}\frac{q_{\parallel} L}{\kappa_{0,e}}$:
\begin{align}
    2n_t T_t &= n_u T_u \nonumber\\
    n_t &= \frac{n_u T_u}{2 T_t}\nonumber\\
    \frac{T_u}{T_t} &= \frac{\left( \frac{7}{2}\frac{q_{\parallel} L}{\kappa_{0,e}} \right)^{2/7}}{ \frac{m_i}{2e}\frac{4q_{\parallel}^2}{\gamma^2e^2n_u^2\left( \frac{7}{2}\frac{q_{\parallel}L}{\kappa_{0e}} \right)^{(4/7)}} }\nonumber\\
\label{eq:stangeby_2point_nt_qpar_nu}
    n_t &= \frac{\gamma^2e^3}{4 m_i}\frac{n_u^3}{q_{\parallel}^2 }\left( \frac{7}{2}\frac{q_{\parallel} L}{\kappa_{0,e}} \right)^{6/7}
\end{align}

\subsubsection{Particle flux density and recycling rate}

The particle flux density onto the target $\Gamma_t$ can be expressed by:
\begin{align}
    \Gamma_t = \frac{q_{\parallel}}{\gamma e T_t} \nonumber\\
    \Gamma_t = \frac{2e}{m_i}\frac{\gamma^2e^2n_u^2 \left( \frac{7}{2}\frac{q_{\parallel}L}{\kappa_{0e}} \right)^{4/7}}{ 4 \gamma e q_{\parallel} } \nonumber\\
\label{eq:stangeby_2point_particle_flux_target}
    \Gamma_t =  \frac{n_u^2}{q_{\parallel}} \left( \frac{7}{2}\frac{q_{\parallel}L}{\kappa_{0e}} \right)^{4/7} \frac{\gamma e^2 }{ 2 m_i}
\end{align}

thus the target density is proportional to:
\begin{equation}
    n_t \propto \frac{n_u^3 L^{6/7}}{q_{\parallel}^{8/7}}
\end{equation}

where it is notable that the target density has a cubic relation to upstream density.

\subsubsection{Sputtering}

The sputtering can be related to the particle flux onto the target by a linear relation with a sputtering function $Y$ representing the amount of sputtering that follows form the particle flux onto the target $\Gamma_t$:
\begin{equation}
    \Gamma_{sput} = Y \Gamma_t
\end{equation}

where the function $Y$ is called the sputtering yield and usually depends on the impact energy, i.e. the target temperature $T_t$. We have both physical sputtering, sputtering due to collisions with the walls/divertor plates, and chemical sputtering, sputtering due to chemical reactions with the walls. Usually chemical sputtering originates from the chemical reactions involving hydrocarbons since the material of the wall are made out of graphite.

For physical sputtering $Y$ can be a strong function of the impact energy, typically $Y \propto T_t^{m} \implies Y \propto n_t^{-m}, m \geq 2$. Thus the sputtering increases with decrease in measurable $\Bar{n}_e$ and vice versa.

For chemical sputtering $Y$ can be a weak function of the impact energy. If the sputtering yield as a function of target temperature $Y(T_t)$ varies less than linearly it results in the opposite effect - increasing $\Bar{n}_e$ will increase the sputtering. For the regime where the sputtering yield function has this property, chemical sputtering might introduce problems. In general sputtering is not wanted.

The strong dependence on the upstream is notable and this should justify expecting a recycling rate $\phi_{recyc} \propto \Bar{n}_e$. Since the particle confinement time is defined as:
\begin{align}
    \tau_p = \frac{\Bar{n}_e \times \textit{volume}}{\phi_{recyc}}\\
    \tau_p \propto \Bar{n}_e^{-1}
\end{align}



\subsection{Extension of the two-point model}

\begin{figure}[H]
    \centering
    \includegraphics[scale=0.5]{Images/Introduction/n_u_n_LCFS_n_t.png}
    \caption{Relating the upstream density, $n_u$, of 1D parallel-to-$\bm B$ analysis, and the $n_{LCFS}$ of 1D radial analysis. The separatrix constitutes the last closed flux surface, LCFS}
    \label{fig:n_u.n_LCFS.n_u}
\end{figure}

\begin{enumerate}
    \item Volumetric power losses and losses due to radiation; Introduced by introducing a power loss factor
    \item Volumetric power losses assumed to occur below the X-point
    \item As long as the ionization zone does not occupy a large fraction of the SOL, the effect on the density and temperature at the target etc. is small.
    \item Similar argument for the spatial radiation distribution and radiation-charge exchange loss
    \item The plasma flow can experience momentum loss by: Friction collisions with neutrals, Viscous forces, Volume recombination
\end{enumerate}

The two-point model is extended by introduction loss factors which represents corrections due to losses:
\begin{enumerate}
    \item $f_{power}$ is the power loss factor. Here it is assumed that the power loss in the SOL can be summed up into two terms; the radiation loss and loss due to charge exchange. Normally the former is larger than the latter.
    \item $f_{mom}$ is the momentum loss factor due to frictional collisions with neutrals, viscous effects and volume recombination.
    \item $f_{cond}$ is the conduction factor representing the correction to the heat conduction due to convection which has the tendency to flatten the temperature gradient.
\end{enumerate}

\subsubsection{Power loss}
Due to radiation and charge exchange there is power loss of the heat entering the SOL:
\begin{equation}
    q_{rad}^{SOL} + q_{cx}^{SOL} \equiv f_{power} q_{\parallel}
\end{equation}

thus the corrected heat flux at the target:
\begin{equation}
    (1 - f_{power})q_{\parallel} = q_t = \gamma k T_t n_t c_{st}
\end{equation}

where $c_{st}$ is the sound speed entering the sheath and it is assumed that the main power loss takes place below the X-point implying that the power loss curve due to radiation is steep. It is similar to the argument that the ionization zone does not need to be vanishingly thin as long as it does not occupy a large fraction of the SOL length.

\subsubsection{Momentum loss factor}

The correction for momentum loss can be expressed as:
\begin{equation}
    p_t = \frac{1}{2}f_{mom}p_u
\end{equation}

because momentum is lost as the particle travels downstream. Here it is assumed that $M_t = 1$, the mach number at the target, originating from the Bohm boundary condition. This corresponds to:
\begin{equation}
    n_t T_t = \frac{1}{2}f_{mom}n_u T_u
\end{equation}

\subsubsection{Conduction correction factor}

Including the flatting of the temperature gradient and thus the heat conduction profile along field lines due to convection the relation between upstream and target temperature can be rewritten as:
\begin{equation}
    T_u = T_t + \frac{7}{2}\frac{f_{cond} q_{\parallel}L}{\kappa_{0,e}}
\label{eq:stangeby_eq_5.23}
\end{equation}

where the power loss factor is excluded as follows from the assumption of power loss below the X-point. Thus Eq. \ref{eq:stangeby_eq_5.23} holds for most of the SOL length $L$ - the volumetric power loss and the target power loss are considered for approximately the same distance from the upstream end. In the Eq. \ref{eq:stangeby_eq_5.23} it is implied that the effect of convection is uniform over most of the SOL length $L$. From the simple two-point model it can be shown that the effect of strong convection close to the target, i.e. from the ionization front to the target, does not affect the target temperature much. For sonic flow the heat conduction from convection is strong since $q_{conv} = 6kT\Gamma$ and $q_{\parallel} = q_t \approx 7kT_t\Gamma_t$ at a value of $T$ slightly above $T_t$. The temperature will not change much when including the convective heat transport from the ionization front to the target. In other words, when the connection length is decreased the effect on the target temperature not large as $T_t \propto L^{-4/7}$.

\subsubsection{The effects of the correction}

\begin{tcolorbox}
For the extended two-point model we have following three equations with three unknowns $n_t, T_t, T_u$:
\begin{align}
    2n_t T_t &= f_{mom}n_u T_u\\
    T_u^{7/2} &\simeq \frac{7}{2}\frac{f_{cond}q_{\parallel}L}{\kappa_{0e}}\\
    (1 - f_{power})q_{\parallel} &= \gamma n_t k T_t c_{st},
\end{align}
\label{eq:ext_two_point_model_stangeby}
\end{tcolorbox}

\textbf{Upstream temperature $T_u$}

When the upstream temperature $T_u$ is assumed to be slightly larger than $T_t$ we get the resulting equations (\ref{eq:T_u_correction_simeq}, \ref{eq:T_u_correction_prop}):
\begin{align}
    T_u &\simeq \left( \frac{7}{2} \frac{f_{cond}q_{\parallel}L}{\kappa_{0e}} \right)^{2/7} \label{eq:T_u_correction_simeq}\\
    T_u &\propto f_{cond}^{2/7} 
\label{eq:T_u_correction_prop}
\end{align}

The upstream temperature is not affected by momentum loss or volumetric power loss, which is only slightly affected by the upstream convection. This results in $f_{cond} < 1$, thus the convection tends to decrease the upstream temperature slightly.

\textbf{Target temperature $T_t$}

Starting with Eq. \ref{eq:T_t_two_point_model_indep_Tu} $T_t$ is related to the correction factors:
\begin{align*}
    2n_t T_t &= f_{mom}n_u T_u\\
    q_{\parallel} &= \frac{1}{1-f_{power}}\gamma n_t k T_t c_{st}\\
    q_{\parallel} &= \frac{1}{2(1-f_{power})}\gamma f_{mom} n_u T_u k \left(\frac{2kT_t}{m_i}\right)^{1/2}\\
    q_{\parallel}^2 &= \left( \frac{1}{2(1-f_{power})}\gamma f_{mom} f_{mom} n_u T_u k \right)^2 \left(\frac{2kT_t}{m_i}\right) = \frac{1}{2(1-f_{power})^2}\gamma^2 f_{mom}^2 n_u^2T_u^2 k^3 \frac{T_t}{m_i}\\
    T_t &= \frac{2(1-f_{power})^2q_{\parallel} m_i}{\gamma^2 f_{mom}^2 n_u^2T_u^2 k^3}\\
    T_t &\simeq \frac{2(1-f_{power})^2q_{\parallel} m_i}{\gamma^2 f_{mom}^2 n_u^2 k^3}\left( \left( \frac{7}{2} \frac{f_{cond}q_{\parallel}L}{\kappa_{0e}} \right)^{2/7} \right)^{-2}\\
    T_t &\propto \frac{(1-f_{power})^2}{f_{mom}^2 } f_{cond}^{-4/7}
\end{align*}

and thus the relation between the target temperature and the correction factors is given in Eq. \ref{eq:stangeby_2point_ex_T_t}:
\begin{equation}
    T_t \propto \frac{(1-f_{power})^2}{f_{mom}^2 f_{cond}^{4/7}}
\label{eq:stangeby_2point_ex_T_t}
\end{equation}

to find the relation between the upstream and target temperature expressed by the correction factors the result from Eq. \ref{eq:T_u_correction_prop} is combined with Eq. \ref{eq:stangeby_2point_ex_T_t} which gives:
\begin{equation}
    \frac{T_u}{T_t} \propto \frac{f_{cond}^{2/7}}{\frac{(1-f_{power})^2}{f_{mom}^2 f_{cond}^{4/7}}} = \frac{f_{mom}^2 f_{cond}^{6/7}}{(1-f_{power})^2}
\label{eq:stangeby_frac_Tt_Tu_corr_fact}
\end{equation}

and this result can be used to express the target density in the correction factors:
\begin{align}
    2n_t T_t &= f_{mom}n_u T_u \nonumber\\
    n_t &\propto f_{mom}\frac{T_u}{T_t} \nonumber\\
\label{eq:stangeby_nt_corr_fact}
    n_t &\propto \frac{f_{mom}^3 f_{cond}^{6/7}}{(1-f_{power})^2}
\end{align}

\subsection{Atomic processes governing the power transport}

The following atomic processes can play a key role in the energy balance in the divertor:
\begin{enumerate}
    \item Impurity radiation
    \item Hydrogen radiation
    \item Ionisation
    \item Charge Exchange collisions
    \item Elastic scattering
    \item Radiation transport
    \item Sputtering
\end{enumerate}

\subsubsection{Productions of impurities}
\textbf{By ion impact}

transport to main plasma

\textbf{By neutral impact}

transport to main plasma

\subsubsection{Impurity Radiation}

Holger: Trade-off between choosing impurity radiation species because a low Z number radiatiates in the wanted temperature range, but the total radiation per species is low compared to higher Z number species

Impurity radiation are due to the emission of photons during the radiative decay of excited states of the various charge states of impurity ions. The distribution of charge states are determined by transport, ionization and recombination (\cite{post1995review}) can by modeled by Eq. \ref{eq:cont_eq_charge_states_post-1995}:
\begin{align}
    \frac{\partial n_z^{+i}}{\partial t} + \nabla \cdot \Gamma_z^{+i} &= n_e n_z^{+i-1}\langle \sigma\nu \rangle_{ionis}^{+i - 1 \to i} - n_e n_z^{+i}( \langle \sigma\nu \rangle_{ionis}^{+i \to i + 1} +\langle \sigma\nu \rangle_{recomb}^{+i \to i - 1})\nonumber\\
    \label{eq:cont_eq_charge_states_post-1995}
    &+ n_e n_z^{+i+1}\langle \sigma\nu \rangle_{recomb}^{+i + 1 \to i}
\end{align}

where $\Gamma_z^{+i}$ is the parallel ion flux for element $Z$ with charge state $+i$, ground state $g$, and $E_l$ is the energy of level $l$. This leads to the equation for the total radiated power (\ref{eq:tot_rad_power_post-1995}):
\begin{equation}
    P_{rad} = \sum_{i=0}^z n_e n_z^{+i}\sum^{all\hspace{0.1cm}l}\langle \sigma \nu \rangle_{excitation}^{g\to l}(E_l - E_g)
\label{eq:tot_rad_power_post-1995}
\end{equation}

The steady state with no transport, called coronal equilibrium, the charge state distribution and the radiation rate coefficient are only dependent on the electron temperature.

\textbf{Impurity radiation from the divertor plasma}

The assessment of conditions needed to radiate the required energy in the divertor can be done by looking at parallel heat conduction and using pressure balance along field lines given in Eqs. \ref{eq:post_1998_par_heat_cond_1} and \ref{eq:post_1998_par_heat_cond_2}:
\begin{align}
    \label{eq:post_1998_par_heat_cond_1}
    \frac{\partial Q_{\parallel}}{\partial r} = -n_e n_Z L_Z(T_e), Q_{\parallel} =& -\kappa_0 T_e^{2.5}\frac{\partial T_e}{\partial x}\\
    p_e =& n_e T_e \implies \frac{1}{2}\frac{\partial }{\partial x}(Q_{\parallel}^2) = \frac{p_e^2}{T_e^2}\kappa_0f_ZL_ZT_e^2{2.5}\frac{\partial T_e}{\partial x}\nonumber\\
    \approx& p_e^2\kappa_0f_ZL_ZT_e^{0.5}\frac{\partial T_e}{\partial x}\nonumber\\
    \implies& \frac{1}{2}Q_{\parallel}^2 \approx p_e^2\kappa_0f_ZL_ZT_e^{0.5}dT_e\nonumber\\
    \label{eq:post_1998_par_heat_cond_2}
    \implies& \frac{\Delta Q_{\parallel}}{n_{es}\sqrt{F_Z}} \approx \sqrt{2 \Bar{\kappa_0}T_{es}^2\int_0^{T_{es}}L_Z(T_e)T_e^{0.5}dT_e}
\end{align}

where $x$ is the direction along field lines, and:
\begin{align*}
    \kappa_0 \approx \frac{3.1\times 10^{9}}{Z_{eff}\ln \Lambda}\left( \frac{\mathrm{erg}}{\mathrm{cm}\mathrm{s}\mathrm{eV}^{3.5}} \right)\\
    F_Z(f_z) = \frac{f_Z(\%)}{Z_eff} = \frac{f_Z(\%)}{1 + 0.01f_Z(\%)Z(Z-1)}\\
    \Bar{\kappa_0} = \kappa Z_{eff}
\end{align*}

\subsubsection{Hydrogen radiation}

For densities $n_e \geq 10^{19}\mathrm{m^{-3}}$, the time between excitation of electrons, de-excitation and ionization collisions are comparable to the radiative decay time of the excited states of hydrogen. This could make "multi-step" processes important where the rate is density dependent. For low temperatures $T_e \leq 3 \mathrm{eV}$ could be important as well. Sputtering can be severely reduced when the temperature is reduced by ionization losses. However, this does not necessarily reduce the peak heat flux since the plasma deposits the ionization energy, $13.6 \mathrm{eV}$ per electron-ion pair for hydrogen, on the divertor plate as it recombines. 

\subsubsection{Recycling}

The recycling of hydrogen atoms and molecules can play a substantial role in the energy transport. The flux of ions along field lines is determined by ionization and recombination in the continuity equation (\cite{post1995review}):
\begin{equation}
    \frac{\partial (n v_{\parallel})}{\partial x} = \langle \sigma \nu \rangle_{ionisation} n_e n_o - \langle \sigma \nu \rangle_{recombination} n_e n_i 
\label{eq:cont_eq_Post-1995}
\end{equation}

The plasma density must be very high and the flow speed very low for three-body and radiative recombination to become important. According to \cite{post1995review} the three-body recombination is important for temperatures $T_e \leq 3 \mathrm{eV}$. The radiation is exploited to spread out and decrease the peak heat flux at the targets. Even though ionization reduces or eliminates the sputtering the ionisation energy ($13.6 \mathrm{eV}$ for each ion-electron pair) is deposited on the target plate as it recombines, thus the peak heat flux is not reduced. However, the calculated amount of hydrogen that ignores "multi-step" effects are significant and in many cases sufficient enough to radiate power from the divertor plasma to the divertor side walls.

\subsubsection{Removal of the $\mathrm{H}$  and $\mathrm{He}$ impurity - pumping}

To avoid fuel dilution the unavoidable $\mathrm{He}$ production from $\mathrm{D}$-$\mathrm{T}$ fusion must be pumped very efficiently as it recycles. According to \cite{stangeby2000plasma} it is probable that helium recycling will exceed the primary ($\mathrm{D}$-$\mathrm{T}$) fusion source. Thus, the pumping of a high fraction $\mathrm{He}$ source is necessary before it re-ionises. 

\subsubsection{Charge exchange}

The charge exchange can happen when a plasma ion collides with a neutral and a transfers the kinetic energy from the fast plasma ion to the neutral. An example is given in Eq. \ref{eq:example_charge_exchange}:
\begin{equation}
    H^{+} + H \Longleftrightarrow H + H^{+}.
\label{eq:example_charge_exchange}
\end{equation}

The net effect is that the hydrogen atom will be fast and the hydrogen ion will be slowed. If the resulting fast hydrogen atom is incident on the plasma the neutral introduces another particle to the plasma. Otherwise it could transfer heat either to the side walls of the reactor or the divertor plate.

The potential of charge exchange is to transfer power from the divertor plasma to the side walls (\cite{post1995calculations}). However, there are many factors at play that affects the effectiveness of this process. Except for temperatures $T_e \leq 3-4 \mathbf{eV}$, the ionization rate is comparable  to the charge exchange rates. 

\subsubsection{Bremsstrahlung losses}

The ratio of losses from bremsstrahlung losses given by Eq. \ref{eq:post_1998_power_loss_bremsstrahlung}:
\begin{equation}
P_{brem} = C_B n_e^2 Z_{eff}T^{1/2},
\label{eq:post_1998_power_loss_bremsstrahlung}
\end{equation}

and alpha heating given by Eq. (for DT fusion): \ref{eq:post_1998_power_loss_alpha}
\begin{equation}
    P_{\alpha} = \frac{n_e^2}{4}f_{DT}^2\langle\sigma \nu \rangle_{DT} E_{\alpha},
\label{eq:post_1998_power_loss_alpha}
\end{equation}

is given by Eq. \ref{eq:post_1998_ratio_ploss_brems_alpha}:
\begin{equation}
    \frac{P_{brem}}{P_{\alpha}} = \frac{4 C_B Z_{eff} T^{1/2}}{f_{DT}^2 \langle \sigma \nu \rangle_{DT} E_{\alpha}}.
\label{eq:post_1998_ratio_ploss_brems_alpha}
\end{equation}

Assumed that $\langle \sigma \nu \rangle_{DT} \approx T^2$ the ratio between bremsstrahlung and alpha heating power losses scales like $\frac{1}{T^{1.5}}$, thus decreasing with the increase in temperature. Further more $\frac{P_{brem}}{P_{\alpha}}$ is proportional to Eq. \ref{eq:post_1998_ratio_ploss_brem_alpha_propto}:
\begin{equation}
    \frac{Z_{eff}}{f_{DT}^2} \approx \frac{1 + 2f_{\alpha} + \sum Z(Z-1)f_Z}{1 - 2f_{\alpha}-\sum Z f_Z}
\label{eq:post_1998_ratio_ploss_brem_alpha_propto}
\end{equation}

where $f_{j} = \frac{n_j}{n_e}$ for any particle of type $j$, thus the magnitude of the ratio  increases with the level of impurities and $Z$.

\subsubsection{Radiation transport}

\textbf{Simple model for radial energy transport at the plasma edge ($r\approx a$)}

Even though edge radiation can be important to reduce the peak heat load on the divertor, there are some potential drawbacks to exhausting all the power by edge radiation according to \cite{post1995calculations}. If the main plasma suffers from large radiation losses the heat flux on the first wall will be increased. Moreover, large amounts of edge radiation reduces the power across the separatrix which could lead to a level below the threshold needed to obtain the H-mode. Cooling the edge may also increase the pumping requirements of Helium exhaust.

These issues can be described by a simplified radial energy transport model given in Eqs. (\ref{eq:post_1998_rad_energy_trans_start}-\ref{eq:post_1998_rad_energy_trans_end}):

\begin{align}
\label{eq:post_1998_rad_energy_trans_start}
    Q_{\perp} &= -\kappa \frac{\partial T}{\partial r}\\
    \frac{\partial Q_{\perp}}{\partial r} &= -n_en_Z L_Z(T_e) \implies Q_{\perp}\frac{\partial Q_{\perp}}{\partial r} = n_e^2f_Z L_Z(T_e)\kappa \frac{\partial T}{\partial r}\\
    \frac{1}{2}\frac{\partial}{\partial r} &= n_e^2f_Z L_Z(T_e)\kappa \frac{\partial T}{\partial r} (Q_{\perp}^2) \\
    \label{eq:post_1998_rad_energy_trans_end}
    &\implies Q_{\perp}^2 = 2\int_0^{T_e}n_e^2 f_z L_Z(T_e)\kappa dT_e.
\end{align}

\subsubsection{Chemical Sputtering}

Hydrogen is very reactive and if graphite walls are used to face the plasma it will produce a flux of hydrocarbons at the plasma edge (\cite{post1995review}). The result is that these hydrocarbons will eventually leave the walls and mix with the edge plasma.
\begin{comment}
\subsubsection{Physical sputtering}

\subsubsection{Molecular collisions}

\end{comment}
section{Elastic scattering of atoms, molecules and ions}

\subsection{Comparison of basic plasma transport properties between stellarator island- and standard tokamak divertors}

