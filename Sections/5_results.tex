\section{Results} 
\label{sec:results}

\todo[inline, color=lightgray!40]
{
    Document the experiments you've run and discuss their outcomes. This includes not only results for the main inference task, but also instrumental objectives such as determining hyperparameters. Make sure that you explain how and why you evaluate your experiments the way you do (metrics, baselines etc.) and present your results with tables and figures wherever appropriate. 
    
    \vspace{0.2cm}
    Please ensure that all tables/figures you produce have descriptive x- and y-axis. Also, if you have more than one attribute shown, include a legend or describe it in the figure caption.
    Further, all tables included in the main part need to be described.
    
    \vspace{0.2cm}
    It might also be reasonable to split up this section. For instance, it could be more natural to first present the experiments and results in a purely objective way, and then add another section with your own subjective interpretation and discussion. 
    
    \vspace{0.2cm}
    Finally, don't be (too) disheartened by negative results. Honestly disclose the outcome of your experiments and analyze them to the best of your ability. If something didn't work out the way you thought, try to figure out why. Based on how much time there is left, you can either try to implement or describe possible solutions.
    
    \vspace{0.2cm}
    \textbf{Amount:} 2 - 4 pages
}
% P_{in} \approx P_{in} = P_{ECRH}
\subsection{$P_{rad}$ Regression with $P_{ECRH}, Z_{eff}, \Bar{n}_e$}

\subsection{$P_{rad}$ Regression with $P_{ECRH}, Z_{eff}, \Bar{n}_e, p_{neutral}$}

\subsubsection{Comment on different ports}

\subsection{$P_{rad}$ Regression with $P_{ECRH}, Z_{eff}, \Bar{n}_e, p_{neutral}, I_{C^{2+}}$}

\subsubsection{Comment on different ports}

\subsection{Collinearity plots}

\subsubsection{Quantify collinearity}

\subsubsection{Comment on collinear variables}

\subsection{Discharges of interest}

\subsubsection{Time evolution of the regression model}